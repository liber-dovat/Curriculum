%% start of file `template_en.tex'.
%% Copyright 2006-1008 Xavier Danaux (xdanaux@gmail.com).
%
% This work may be distributed and/or modified under the
% conditions of the LaTeX Project Public License version 1.3c,
% available at http://www.latex-project.org/lppl/.


\documentclass[11pt,a4paper]{moderncv}
\usepackage[spanish]{babel}
% moderncv themes
\moderncvtheme[blue]{casual}                            % optional argument are 'blue' (default), 'orange', 'red', 'green', 'grey' and 'roman' (for roman fonts, instead of sans serif fonts)
% \moderncvtheme[red]{classic}                          % idem
% character encoding
\usepackage[utf8]{inputenc}                             % replace by the encoding you are using
% adjust the page margins
%\usepackage[scale=0.8]{geometry}
%\setlength{\hintscolumnwidth}{3cm}                     % if you want to change the width of the column with the dates
%\AtBeginDocument{\setlength{\maketitlenamewidth}{6cm}} % only for the classic theme, if you want to change the width of your name placeholder (to leave more space for your address details
\AtBeginDocument{\recomputelengths}                     % required when changes are made to page layout lengths
% personal data
\firstname{Liber}
\familyname{Dovat}
\title{Curriculum vitae}                        % optional, remove the line if not wanted
% \address{}{}                                  % optional, remove the line if not wanted
\mobile{099 109 509}                            % optional, remove the line if not wanted
% \phone{}                                      % optional, remove the line if not wanted
% \fax{fax (optional)}                          % optional, remove the line if not wanted
\email{liber.dovat@gmail.com}                   % optional, remove the line if not wanted
% \extrainfo{additional information (optional)} % optional, remove the line if not wanted
\photo[70pt]{liber_2.eps}                         % '64pt' is the height the picture must be resized to and 'picture' is the name of the picture file; optional, remove the line if not wanted
\quote{Curriculum vitae}                        % optional, remove the line if not wanted
%\nopagenumbers{}                               % uncomment to suppress automatic page numbering for CVs longer than one page


%----------------------------------------------------------------------------------
%            content
%----------------------------------------------------------------------------------
\begin{document}
\maketitle
% \section{Education}
% \cventry{year--year}{Degree}{Institution}{City}{\textit{Grade}}{Description}  % arguments 3 to 6 are optional
% \cventry{year--year}{Degree}{Institution}{City}{\textit{Grade}}{Description}  % arguments 3 to 6 are optional
%
% \section{Master thesis}
% \cvline{title}{\emph{Title}}
% \cvline{supervisors}{Supervisors}
% \cvline{description}{\small Short thesis abstract}
%
% \section{Experience}
% \subsection{Vocational}
% \cventry{year--year}{Job title}{Employer}{City}{}{Description}                % arguments 3 to 6 are optional
% \cventry{year--year}{Job title}{Employer}{City}{}{Description}                % arguments 3 to 6 are optional
% \subsection{Miscellaneous}
% \cventry{year--year}{Job title}{Employer}{City}{}{Description line 1\newline{}Description line 2}% arguments 3 to 6 are optional
%
% \section{Languages}
% \cvlanguage{language 1}{Skill level}{Comment}
% \cvlanguage{language 2}{Skill level}{Comment}
% \cvlanguage{language 3}{Skill level}{Comment}
%
% \section{Computer skills}
% \cvcomputer{category 1}{XXX, YYY, ZZZ}{category 4}{XXX, YYY, ZZZ}
% \cvcomputer{category 2}{XXX, YYY, ZZZ}{category 5}{XXX, YYY, ZZZ}
% \cvcomputer{category 3}{XXX, YYY, ZZZ}{category 6}{XXX, YYY, ZZZ}

\section{Datos personales}
\cvline{Cédula de identidad:}{4.663.221-7}
\cvline{Nacionalidad:}{Uruguayo}
\cvline{Fecha de nacimiento:}{24 de setiembre de 1984}
\cvline{Estado civil:}{Soltero}
%\cvline{Dirección:}{Chapicuy, Manzana 8B Solar 6, El Pinar, Canelones, Uruguay}
% \cvline{Dirección:}{Gonzalo Ramírez 1686 apto. 104, Montevideo, Uruguay}
\cvline{Dirección:}{Rivera 2156 apto. 101, Montevideo, Uruguay}
\cvline{Teléfono:}{099 109 509}
\cvline{Correo electrónico:}{liber.dovat@gmail.com}

%----------------------------

\section{Formación académica}
\cventry{2004 -}{Facultad de Ingeniería, Universidad de la República}{}{Ingeniería en computación}{}{}
\cvline{}{}
\cventry{2000-2002}{Instituto Alfredo Vásquez Acevedo}{}{}{Bachillerato científico, opción Ingeniería}{}

%----------------------------

\section{Experiencia laboral}

\cventry{Julio del 2016 hasta diciembre del 2016}
        {Ayudante}{}
        {Fundación Julio Ricaldoni}{Laboratorio de Energía Solar}
        {Fui contratado por la Fundación Julio Ricaldoni (FJR) para realizar tareas para el Laboratorio de Energía Solar (LES), para colaborar con las tareas de construcción y mantenimiento de la infraestructura informática de dicho laboratorio. En particular, se busca actualizar y modificar el Sistema de Gestión de Datos que funciona en forma online en el sitio web del LES.
        \newline{}
        Entre las tareas a desempeñar podemos resaltar poner a punto el servicio online de Gestión de Datos del LES, automatizar el proceso de carga de datos en el Sistema de Gestión de Datos del LES, colaborar con el mantenimiento de la infraestructura informática del LES y de su sitio web, colaborar con las tareas del LES en la recepción automática de los datos de la Red de Medida Continua de Irradiancia Solar (RMCIS) y en el procesamiento de información satelital, y realizar tareas de soporte informático para el laboratorio.
        }
\cvline{}{}
\cventry{Mayo del 2013 hasta diciembre del 2015}
        {Docente ayudante grado 2}{}
        {Instituto de Física, Facultad de Ingeniería, Universidad de la República}{}
        {En el marco de una extensión horaria a mi cargo como docente, a cargo del \textbf{convenio FING/MIEM ``Mantenimiento de la Red de Estaciones de Medida Continua de Radiación Solar y Generación de Información detallada sobre el Recurso Solar en el Uruguay''} realicé las siguientes tareas:
        \newline{}
        Realicé un estudio que permite ver las imágenes satelitales que se encuentran actualmente a nuestra disposición, y que permite apreciar de forma visual los rangos horarios para los que existen y no existen muestras (ya sea por problemas con el satélite, o debido a modificaciones del cronograma del mismo).
        \newline{}
        Realicé modificaciones del programa de procesamiento de imágenes para poder clasificar entre las imágenes que resultan de interés procesar y las que son descartadas como inválidas. Con Rodrigo Alonso y Nicolás Wainstein estudiamos alternativas de modificación en el código de procesamiento de las imágenes para poder hacer más eficiente dicho algoritmo, y se logró reducir considerablemente el tiempo de procesamiento, reduciéndolo de 70 segundos a 1 segundo aproximadamente.
        \newline{}
        Con Gonzalo Abal, Rodrigo Alonso y Daniel Aicardi ideamos un esquema de base de datos para poder alojar distintos productos obtenidos por el procesamiento de la información adquirida en la estaciones, y también para poder hacer el seguimiento de las distintas estaciones y equipamiento que hay en ellas. Dicha base de datos está mantenida en un servidor \textbf{PostgreSQL}, y ya fueron ingresados todos los datos adquiridos hasta la fecha.
        \newline{}
        Una vez ingresados todos los datos estacionales en la base de datos, se procedió a desarrollar una página web dinámica utilizando el framework \textbf{Slim}, el cual trabaja sobre PHP. A través de dicha página se pueden realizar consultas a la base de datos, pudiendo desplegar el resultado en una tabla web, y que a su vez es posible descargar los resultados como un archivo con formato .csv. La página web se encuentra publicada, y es parte del sitio web \url{les.edu.uy}. Estamos estudiando la posibilidad de incluir más características.
        \newline{}
        Instalé y configuré un servidor \textbf{LAMP} con \textbf{Wordpress}, en donde se migró el sitio web \url{les.edu.uy}, y de esta manera pudimos obtener un servidor propio para alojar dicho sitio.
        }
\cvline{}{}
\cventry{Abril del 2013 hasta la actualidad}
        {Docente ayudante grado 2}{}
        {Instituto de Física, Facultad de Ingeniería, Universidad de la República}{}
        {Actualmente realizo tareas de soporte informático y asistencia técnica a usuarios en una red \textbf{TCP/IP} con autenticación \textbf{LDAP} con plataformas \textbf{Linux}, \textbf{Mac} y \textbf{Windows}, en donde estos últimos están integrados a la red mediante un servidor samba.
         \newline{}
         Administro las cuentas de los usuarios de windows y linux, realizando pedidos a la URI cuando es necesario crear casillas de correo, o la creación y edición de los grupos y alias de correo.
         \newline{}
         Administro las impresoras del instituto, y diagnostico y reparo distintos errores que presentan los equipos, y realizo el seguimiento de los cartuchos de tóner utilizados por éstas y la administración de sus recargas.
         \newline{}
         Estoy a cargo de la compra de insumos referentes a informática y equipos computadores para los docentes y el instituto, de la instalación y configuración del sistema operativo y software que utilizan, de la compra de hardware de repuesto para realizar reparaciones y actualizaciones de los computadores, y hardware de comunicación como switchs y routers necesarios para la conexión en red de los equipos computadores pertenecientes al instituto.
         \newline{}
         Mantengo la infraestructura de la red cableada, y estoy a cargo del relevamiento de las notebooks de los docentes del instituto y de los docentes visitantes, ya que estas se conectan a la red mediante un servidor DHCP administrado por la URI.
         \newline{}
         Instalo y configuro computadores con sistemas operativos Windows y GNU/Linux. Los equipos son instalados con el software de uso estándar como las suites de oficina, navegadores de internet y clientes de correo. En su gran mayoría los equipos fueron instalados de forma individual, pero cuando la cantidad de equipos era considerable y de características similares usé el método de clonación de discos para realizar la tarea mucho más rápido y de forma más eficiente.
         }
\cvline{}{}
\cventry{Febrero del 2008 - abril del 2013}
        {Docente ayudante grado 1}{}
        {Instituto de Física, Facultad de Ingeniería, Universidad de la República}{}
        {Además de realizar tareas de soporte informático y asistencia técnica que realizo actualmente, realicé las siguentes tareas:
         \newline{}
         Con Ricardo Siri migramos los equipos GNU/Linux que utilizaban el sistema NIS de autenticación al nuevo sistema LDAP.
         \newline{}
         Rediseñé la portada de la página web del instituto utilizando HTML y CSS, y realicé tareas de mantenimiento y actualizaciones en dicha página web, además de la creación de las páginas web del grupo de Física Computacional y el grupo de investigación de Solarimetría, y brindé soporte para la creación y mantenimiento de la página web para la sexta conferencia de Óptica Cuántica, la cual fue creada utilizando HTML5 y CSS3.
         \newline{}
         Migré la página el instituto al nuevo sistema de manejo de contenidos \textbf{Drupal}, y migré a dicho sistema las páginas de los distintos grupos de investigación del instituto.
         \newline{}
         Brindé soporte y realicé el seguimiento de la compra e instalación de los porteros eléctricos del instituto.
         \newline{}
         Integré la comisión encargada de seleccionar el logotipo del instituto.
         }
\cvline{}{}
\cventry{Julio del 2006 - julio del 2007}
        {Docente ayudante grado 1}{}
        {Instituto de Ingeniería Mecánica y Producción Industrial, Facultad de Ingeniería, Universidad de la República}{}
        {Realicé tareas de soporte informático y asistencia técnica a usuarios en ambientes con \textbf{Windows 2000} y \textbf{Windows XP}, administré y coordine la utilización de los equipos para uso docente tales como Proyectores y Laptops, y actualización y mantenimiento de la página web de los cursos del instituto.
         \newline{}
         Con el equipo de trabajo realizamos la completa migración de los equipos del instituto que utilizaban Windows 2000 a Windows XP.
         \newline{}
         Con Mario Madera rediseñamos la portada de la página web del instituto utilizando HTML y CSS.
        }

%----------------------------

\section{Idiomas}
\cvline{Francés}{}
\cvline{2015}{Aprobado el curso de \textbf{Francés 1} en la Facultad de Humanidades de la Universidad de la República.}
\cvline{}{}
\cvline{Japonés}{}
\cvline{2014}{Aprobado el examen internacional de aptitud del idioma \textbf{Japonés} nivel \textbf{N4} (Nihongo N\={o}ryoku Shiken N4).}
\cvline{2013}{Aprobado el curso de \textbf{Japonés 3} en la Facultad de Humanidades de la Universidad de la República.}
\cvline{2012}{Aprobado el examen internacional de aptitud del idioma \textbf{Japonés} nivel \textbf{N5} (Nihongo N\={o}ryoku Shiken N5).}
\cvline{}{}
\cvline{Inglés}{Buen manejo a nivel intermedio y técnico.}
% \cvline{Japonés}{2013 - Aprobado el curso de \textbf{Japonés 3} en la Facultad de Humanidades de la
%                         Universidad de la República.
%                  \newline{}
%                  2012 - Aprobado el examen internacional de aptitud del idioma japonés nivel N5 (Nihongo Nouryoku Shiken N5).
%                  \newline{}
%                  2014 - Aprobado el examen internacional de aptitud del idioma japonés nivel N4 (Nihongo Nouryoku Shiken N4).
%                 }

%----------------------------

\section{Conocimientos en informática}

\cvline{}{}
\cvline{Modelos de desarrollo}
       {Conocimientos en el modelo de desarrollo ágil \textbf{SCRUM}.
       }
\cvline{}{}
\cvline{Tecnologías de integración}
       {Configuración y puesta en marcha de aplicaciónes en los entornos \textbf{Heroku} y \textbf{Amazon Web Services (AWS)}, utilizando como software de Integración continua la aplicación \textbf{Jenkins}.
       }
\cvline{}{}
\cvline{Programación}
       {Conocimientos de programación en los lenguajes \textbf{Ruby} utilizando el framework \textbf{Rails}, \textbf{Java}, \textbf{C++}, \textbf{C}, \textbf{Perl}, \textbf{Python}, \textbf{Pascal}, \textbf{Modula2}, \textbf{SDL}, \textbf{OpenGL}.
       }
\cvline{}{}
\cvline{Bases de Datos}
       {Conocimientos de realización de consultas en las bases de datos \textbf{PostgreSQL} y \textbf{MySQL}, así como creación, edición y mantenimiento de dichas bases de datos.
       }
\cvline{}{}
\cvline{Sistemas operativos}
       {Instalación, configuración y administración de sistemas operativos de la rama \textbf{GNU/Linux} (Arch linux, openSuSE, Ubuntu, Gentoo, Fedora), sistemas \textbf{Windows} (98, 2000, XP, Vista, 7), y configuración de sistemas \textbf{Mac OS X}.
       }
\cvline{}{}
\cvline{Servidores Web}
       {Instalación y configuración de servidores tipo \textbf{LAMP} con \textbf{Wordpress}.
       }
\cvline{}{}
\cvline{Ofimática}
       {\LaTeX{}, LibreOffice/OpenOffice y Microsoft Office.
       }
\cvline{}{}
\cvline{Diseño gráfico}
       {Creación y edición de imágenes utilizando \textbf{The Gimp}, \textbf{Adobe Photoshop} y \textbf{Corel PhotoPaint}.
        \newline{}
        Conocimientos avanzados de dibujo vectorial y multimedia en \textbf{Inkscape (SVG)}, \textbf{Macromedia Flash MX}, \textbf{Macromedia Fireworks} y \textbf{Corel Draw}.
        \newline{}
        Diseño y composición de escenas, entornos y elementos en 3D utilizando \textbf{Blender}.
       }
\cvline{}{}
\cvline{Desarrollo web}
       {Creación y edición de páginas web utilizando \textbf{HTML5}, \textbf{CSS3}, \textbf{Dreamweaver} y el sistema de manejo de contenidos \textbf{Drupal}.
       \newline{}
       Conocimientos avanzados en \textbf{PHP}, \textbf{Javascript} y el framework de desarrollo \textbf{Slim}.
       }
\cvline{}{}
\cvline{Internet}
       {Nivel avanzado en movimientos y consultas en Internet: Búsqueda de información, recursos de software, etc\dots
       }

%----------------------------

\section{Actividades de investigación}
\cventry{2014}{II\textsuperscript{as} Jornadas de Cartografía}{}{junto con unos compañeros de la Facultad de Ingeniería de la UdelaR, fuimos ponientes con el trabajo ``Consultas                            
               Espaciales Multifuente en el Contexto IDEuy''}
              {}
              {
               \textbf{Resumen:}\\
% \small
% \footnotesize
% \scriptsize
               En el marco de la asignatura Taller de Sistemas de Información Geográficos Empresariales (TSIGE)
               dictado en la Facultad de Ingeniería de la UdelaR se platea la realización de un proyecto que
               promueve la investigación de tecnologías SIG siendo el presente artículo el resultado de dicha
               investigación.\\
               Este proyecto tiene como finalidad investigar los servicios ofrecidos por parte de
               \href{http://www.agesic.gub.uy/innovaportal/v/665/1/agesic/IDE.html}{IDEuy},
               concentrando la atención en los servicios de tipo Web Feature Service (WFS).
               Luego, a partir de los servicios más completos, se decide utilizar el provisto por la Intendencia
               de Montevideo, para luego generar varios casos de estudio, las cuales utilizan fuentes de datos
               presentes en una base de datos local, y fuentes externas que pertenecen al IDEuy y son administradas
               por el organismo que corresponda. A partir de dichos casos de estudio se desarrolló una aplicación
               web en donde se pueden realizar las consultas y en donde cada una de ellas acceda a una fuente de
               datos local y otra externa al sistema.
              }

%----------------------------

\section{Otras actividades}
\cventry{2000}{Digisoft}{}{Aprobado el curso de Usuario Windows}{}{}

%----------------------------

\section{Intereses personales}
\cvline{\textbullet}{Interesado en la filosofía del software libre.}
\cvline{\textbullet}{Creación de interfaces de software utilizando las bibliotecas gráficas \textbf{Qt}, \textbf{SDL} y \textbf{GTK}.}
\cvline{\textbullet}{Fotografía.}
\cvline{\textbullet}{Dibujo digital.}
\cvline{\textbullet}{Ampliar mis conocimientos en idiomas, en particular Alemán, Italiano y Francés.}

\end{document}

